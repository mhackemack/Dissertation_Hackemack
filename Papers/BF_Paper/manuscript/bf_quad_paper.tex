%%
%% This is file `elsarticle-template-num.tex',
%% generated with the docstrip utility.
%%
%% The original source files were:
%%
%% elsarticle.dtx  (with options: `numtemplate')
%% 
%% Copyright 2007, 2008 Elsevier Ltd.
%% 
%% This file is part of the 'Elsarticle Bundle'.
%% -------------------------------------------
%% 
%% It may be distributed under the conditions of the LaTeX Project Public
%% License, either version 1.2 of this license or (at your option) any
%% later version.  The latest version of this license is in
%%    http://www.latex-project.org/lppl.txt
%% and version 1.2 or later is part of all distributions of LaTeX
%% version 1999/12/01 or later.
%% 
%% The list of all files belonging to the 'Elsarticle Bundle' is
%% given in the file `manifest.txt'.
%% 

%% Template article for Elsevier's document class `elsarticle'
%% with numbered style bibliographic references
%% SP 2008/03/01

%\documentclass[preprint,12pt]{elsarticle}
\documentclass[preprint,10pt]{elsarticle}
%\documentclass[final,3p,times]{elsarticle} 

%% Use the option review to obtain double line spacing
%% \documentclass[authoryear,preprint,review,12pt]{elsarticle}

%% Use the options 1p,twocolumn; 3p; 3p,twocolumn; 5p; or 5p,twocolumn
%% for a journal layout:
%% \documentclass[final,1p,times]{elsarticle}
%% \documentclass[final,1p,times,twocolumn]{elsarticle}
%% \documentclass[final,3p,times]{elsarticle}
%% \documentclass[final,3p,times,twocolumn]{elsarticle}
%% \documentclass[final,5p,times]{elsarticle}
%% \documentclass[final,5p,times,twocolumn]{elsarticle}

%% if you use PostScript figures in your article
%% use the graphics package for simple commands
\usepackage{subfigure}

\usepackage{color}
%% or use the graphicx package for more complicated commands
\usepackage{graphicx}
%% or use the epsfig package if you prefer to use the old commands
%% \usepackage{epsfig}

%% The amssymb package provides various useful mathematical symbols 
%% The amsthm package provides extended theorem environments
\usepackage{amssymb}
\usepackage{amsmath}
% more math
\usepackage{amsfonts}
\usepackage{amstext}
\usepackage{amsbsy}
%\usepackage{mathbbol} 

%% The lineno packages adds line numbers. Start line numbering with
%% \begin{linenumbers}, end it with \end{linenumbers}. Or switch it on
%% for the whole article with \linenumbers.
\usepackage{lineno}

\journal{Journal of Comp. Phys.}
%%%%%%%%%%%%%%%%%%%%%%%%%%%%%%%%%%%%%%%%%%%%%%%%%%%%%%%%%%%%%%%%%%%%
% operators
\renewcommand{\div}{\vec{\nabla}\! \cdot \!}
\newcommand{\grad}{\vec{\nabla}}
% latex shortcuts
\newcommand{\bea}{\begin{eqnarray}}
\newcommand{\eea}{\end{eqnarray}}
\newcommand{\be}{\begin{equation}}
\newcommand{\ee}{\end{equation}}
\newcommand{\bal}{\begin{align}}
\newcommand{\eali}{\end{align}}
\newcommand{\bi}{\begin{itemize}}
\newcommand{\ei}{\end{itemize}}
\newcommand{\ben}{\begin{enumerate}}
\newcommand{\een}{\end{enumerate}}
% DGFEM commands
\newcommand{\jmp}[1]{[\![#1]\!]}                     % jump
\newcommand{\mvl}[1]{\{\!\!\{#1\}\!\!\}}             % mean value
\newcommand{\keff}{\ensuremath{k_{\textit{eff}}}\xspace}
% shortcut for domain notation
\newcommand{\D}{\mathcal{D}}
% vector shortcuts
\newcommand{\vo}{\vec{\Omega}}
\newcommand{\vr}{\vec{r}}
\newcommand{\vn}{\vec{n}}
\newcommand{\vnk}{\vec{\mathbf{n}}}
\newcommand{\vj}{\vec{J}}
\newcommand{\eig}[1]{\| #1 \|_2}

\newcommand{\EI}{\mathcal{E}_h^i}
\newcommand{\ED}{\mathcal{E}_h^{\partial \D^d}}
\newcommand{\EN}{\mathcal{E}_h^{\partial \D^n}}
\newcommand{\ER}{\mathcal{E}_h^{\partial \D^r}}
\newcommand{\reg}{\textit{reg}}

% extra space
\newcommand{\qq}{\quad\quad}
% common reference commands
\newcommand{\eqt}[1]{Eq.~(\ref{#1})}                     % equation
\newcommand{\fig}[1]{Fig.~\ref{#1}}                      % figure
\newcommand{\tbl}[1]{Table~\ref{#1}}                     % table
\newcommand{\sct}[1]{Section~\ref{#1}}                   % section

\newcommand\br{\mathbf{r}}
%\newcommand{\tf}{\varphi}
\newcommand{\tf}{b}

\newcommand{\tcr}[1]{\textcolor{red}{#1}}
\newcommand{\mt}[1]{\marginpar{ {\tiny #1}}}
\newcommand{\Introfigpath}[1]{../../../Document/Rev0/figures/sec_Intro/{#1}}
\newcommand{\Snfigpath}[1]{../../../Document/Rev0/figures/sec_Sn/{#1}}
\newcommand{\BFfigpath}[1]{../../../Document/Rev0/figures/sec_BF/{#1}}
%%%%%%%%%%%%%%%%%%%%%%%%%%%%%%%%%%%%%%%%%%%%%%%%%%%%%%%%%%%%%%%%%%%%%
%
%   BEGIN DOCUMENT
%
%%%%%%%%%%%%%%%%%%%%%%%%%%%%%%%%%%%%%%%%%%%%%%%%%%%%%%%%%%%%%%%%%%%%%
\begin{document}

 

%%%%%%%%%%%%%%%%%%%%%%%%%%%%%%%%%%%%%%%%%%%%%%%%%%%%%%%%%%%%%%%%%%%%
\begin{frontmatter}

%% Title, authors and addresses

%% use the tnoteref command within \title for footnotes;
%% use the tnotetext command for theassociated footnote;
%% use the fnref command within \author or \address for footnotes;
%% use the fntext command for theassociated footnote;
%% use the corref command within \author for corresponding author footnotes;
%% use the cortext command for theassociated footnote;
%% use the ead command for the email address,
%% and the form \ead[url] for the home page:
%\title{Title\tnoteref{label1}}
%% \tnotetext[label1]{}
%% \author{Name\corref{cor1}\fnref{label2}}
%% \ead{email address}
%% \ead[url]{home page}
%% \fntext[label2]{}
%% \cortext[cor1]{}
%% \address{Address\fnref{label3}}
%% \fntext[label3]{}

%-------------------------
%-------------------------
%\title{Discontinuous Finite Element Solution for Diffusion Equations on Arbitrary Polygonal Meshes}
%\title{Higher-Order Basis Functions for the DGFEM $S_N$ Transport Equation on Arbitrary Polygonal Grids}
\title{Higher-Order Discontinuous Finite Element Discretization for $S_N$ Transport on Arbitrary Polygonal Grids}
%-------------------------
%-------------------------

%% use optional labels to link authors explicitly to addresses:
%% \author[label1,label2]{}
%% \address[label1]{}
%% \address[label2]{}

%-------------------------
\author{Michael W. Hackemack}
\author{Jean C. Ragusa}
\ead{jean.ragusa@tamu.edu}
\address{Department of Nuclear Engineering, Texas A\&M University, College Station, TX 77843, USA}

% \cortext[cor1]{Corresponding author}
%-------------------------

%-------------------------
\begin{abstract}

abstract goes here...
 

\end{abstract}
%-------------------------

%-------------------------
\begin{keyword}
  Radiation Transport \sep
	Arbitrary Polygonal Grids \sep
  Discontinuous Finite Element \sep
	Higher-Order Basis Functions
\end{keyword}
%-------------------------

\end{frontmatter}

%%%%%%%%%%%%%%%%%%%%%%%%%%%%%%%%%%%%%%%%%%%%%%%%%%%%%%%%%%%%%%%%%%%%

\linenumbers

%%%%%%%%%%%%%%%%%%%%%%%%%%%%%%%%%%%%%%%%%%%%%%%%%%%%%%%%%%%%%%%%%%%%
%%%%%%%%%%%%%%%%%%%%%%%%%%%%%%%%%%%%%%%%%%%%%%%%%%%%%%%%%%%%%%%%%%%%
\section{Introduction} \label{sec::intro}
%%%%%%%%%%%%%%%%%%%%%%%%%%%%%%%%%%%%%%%%%%%%%%%%%%%%%%%%%%%%%%%%%%%%
%%%%%%%%%%%%%%%%%%%%%%%%%%%%%%%%%%%%%%%%%%%%%%%%%%%%%%%%%%%%%%%%%%%%
%\begin{itemize}
%\item 
%\item 
%\item 
%\end{itemize}

\tcr
{
\begin{enumerate}
\item introduce continuous, 1G transport equation
\begin{itemize}
\item application of TE (reactors, HDELP, astrophy, medical,...)
\item oin this paper, we focus on solution techniques based on first-order form of the TE, namely, Sn DO in angle and DFEM in space. Why? The spatial disc of Sn transport dates back to Lessain/Raviart 1972, Reed/Hill 1973, and amenable to efficient iterative techniques, matrix-free transport sweeps + DSA precondition. Works well on massively // architectures.  
\item most of DGFEM Sn Transport code (Denovo, Partisn, Capsacin) use linear on simplices and hex grids, except PDT, PWLD Adams on abitrary polyg/h
\item also bring some focus on TDL (thick)
\end{itemize}
\item history of higher-order Sn transport discretizations
\begin{itemize}
\item DGFEM: Reed high-order triangular, Wang (AMR), 
\item nodal method transport (review paper), Azmy AHOT, Wachspress 
\item acknowledge basis functions derived/used here can be employed for other spatial disc of transport on arbitrary grids, SAAF, LS, EP, (CFEM) [some form can be Sn in agle or Pn, e.g., SAAF-PN/SN, EVENT UK Imperial  + RATTLESNAKE). Important to recall this in conclusions. 
\end{itemize}
\item polygonal grids + poly math
\begin{itemize}
\item HOWEVER, all of the above: nada for polygons except PWLD which is only linear
\item why polyg? 4 reasons 
\item Rand/Gillete sukumar
\end{itemize}
\item outline
\begin{itemize}
\item brief review DG/ Sn
\item brief review of linear basis on polygon
\item finally, the meat: (1) application (new) of existing HO basis function to Transport (2) novelty: PWQ on poly
\end{itemize}
\end{enumerate}
}


% Refined quadrilateral vertices figure
% ---------
\iffalse
\begin{figure}[hbt]
\centering
\includegraphics[width=0.85\textwidth]{\Introfigpath{locally_refined_vertices.png}}
\caption{Local mesh refinement of an initial quadrilateral cell (left) leads to a degenerate pentagonal cell (right) without the use of a hanging node.}
\label{fig::Intro_locally_refined_vertices}
\end{figure}
\fi
% ---------

%%%%%%%%%%%%%%%%%%%%%%%%%%%%%%%%%%%%%%%%%%%%%%%%%%%%%%%%%%%%%%%%%%%%
%%%%%%%%%%%%%%%%%%%%%%%%%%%%%%%%%%%%%%%%%%%%%%%%%%%%%%%%%%%%%%%%%%%%
\section{The DGFEM $S_N$ Discretization of the Transport Equation} \label{sec::dgfem}
%%%%%%%%%%%%%%%%%%%%%%%%%%%%%%%%%%%%%%%%%%%%%%%%%%%%%%%%%%%%%%%%%%%%
%%%%%%%%%%%%%%%%%%%%%%%%%%%%%%%%%%%%%%%%%%%%%%%%%%%%%%%%%%%%%%%%%%%%
\tcr
{ keep it short
\begin{enumerate}
\item angular discretization
\item DGFEM discretization with upwind scheme
\item theoretical convergence rates (including solution irregularity) Wang + others (Pitkarantka ...)
\end{enumerate}
}


%%%%%%%%%%%%%%%%%%%%%%%%%%%%%%%%%%%%%%%%%%%%%%%%%%%%%%%%%%%%%%%%%%%%
%%%%%%%%%%%%%%%%%%%%%%%%%%%%%%%%%%%%%%%%%%%%%%%%%%%%%%%%%%%%%%%%%%%%
\section{Linear Polygonal Basis Functions} \label{sec::linpoly}
%%%%%%%%%%%%%%%%%%%%%%%%%%%%%%%%%%%%%%%%%%%%%%%%%%%%%%%%%%%%%%%%%%%%
%%%%%%%%%%%%%%%%%%%%%%%%%%%%%%%%%%%%%%%%%%%%%%%%%%%%%%%%%%%%%%%%%%%%
\tcr
{
\begin{enumerate}
\item Image of geometric terms on general polygon
\item List/description of barycentric properties
\end{enumerate}
}

%------------------------------------------------------------------------------------------------------------
\subsection{Wachspress Rational Functions}
%------------------------------------------------------------------------------------------------------------

\tcr{short history and functional form}

The first linearly-complete polygonal functions that we will consider are the Wachspress rational functions \cite{wachspress1975rational}. 

\begin{equation}
\label{eq::wach_BF}
\lambda_{j}^{W} (\vec{r}) = \frac{w_j (\vec{r}) }{\sum\limits_{i=1}^{n} w_i (\vec{r})},
\end{equation}

\noindent where the Wachspress weight function for vertex $j$, $w_j$, has the following definition:

\begin{equation}
\label{eq::wach_weights}
w_j (\vec{r})  = \frac{A(\vec{r}_{j-1}, \vec{r}_{j}, \vec{r}_{j+1})}{A(\vec{r}, \vec{r}_{j-1}, \vec{r}_{j}) \, A(\vec{r}, \vec{r}_{j}, \vec{r}_{j+1})} .
\end{equation}

\noindent In Eq. (\ref{eq::wach_weights}), the terms $A(\vec{a}, \vec{b}, \vec{c})$ denote the signed area of the triangle with vertices $\vec{a}$, $\vec{b}$, and $\vec{c}$. Each of these signed areas can be computed by

\begin{equation}
\label{eq::wach_signed_area}
A(\vec{a}, \vec{b}, \vec{c}) = \frac{1}{2}
\left|  
  \begin{array}{ccc}
  1 & 1 & 1 \\
  x_a & x_b & x_c \\
  y_a & y_b & y_c
  \end{array}
\right| .
\end{equation}

%------------------------------------------------------------------------------------------------------------
\subsection{Piecewise Linear Functions}
%------------------------------------------------------------------------------------------------------------

\tcr{short history and functional form}

\begin{equation}
\label{eq::PWL_2D}
\lambda_j^{PWL} (\vec{r}) = t_j (\vec{r}) + \alpha_j^K t_c (\vec{r}) .
\end{equation}

\noindent In Eq. (\ref{eq::PWL_2D}), $t_j$ is the standard 2D linear function with unity at vertex $j$ that linearly decreases to zero at the cell center and each adjoining vertex. $t_c$ is the 2D cell ``tent'' function located at $\vec{r}_{c}$ which is unity at the cell center and linearly decreases to zero at each cell vertex. $\alpha_{j}^{K}$ is the weight parameter for vertex $j$ in cell $K$. The functional form of Eq. (\ref{eq::PWL_2D}) with identical vertex weights means that the PWL function for vertex $j$, within the domain of $K$, linearly decreases to a value of $1/n$ at the polygonal center. From there, the function linearly decreases to zero on all faces that are not connected to vertex $j$.

%------------------------------------------------------------------------------------------------------------
\subsection{Mean Value Coordinates}
%------------------------------------------------------------------------------------------------------------

\tcr{short history and functional form}

\begin{equation}
\label{eq::MV_BF}
\lambda_{j}^{MV} (\vec{r}) = \frac{w_j (\vec{r}) }{\sum\limits_{i=1}^{n} w_i (\vec{r})} ,
\end{equation}

\noindent where the mean value weight function for vertex $j$, $w_j$, has the following definition:

\begin{equation}
\label{eq::MV_weights}
w_j (\vec{r})  = \frac{\tan(\alpha_{j-1} / 2) + \tan(\alpha_j / 2)}{|\vec{r}_j - \vec{r}|}.
\end{equation}

%------------------------------------------------------------------------------------------------------------
\subsection{Maximum Entropy Coordinates}
%------------------------------------------------------------------------------------------------------------

\tcr{short history and functional form}

\begin{equation}
\label{eq::ME_BF}
\lambda_{j}^{ME} (\vec{r}) = \frac{w_j (\vec{r}) }{\sum\limits_{i=1}^{n} w_i (\vec{r})} .
\end{equation}

%------------------------------------------------------------------------------------------------------------
\subsection{Summary of the Linear Polygonal Basis Functions}
%------------------------------------------------------------------------------------------------------------

\tcr
{
\begin{enumerate}
\item table with summary
\item example contour plots (small)
\end{enumerate}
}

\begin{table}
\centering
\caption{Summary of the properties of the 2D coordinate systems used on polygons.}
\begin{tabular}{|c|c|c|c|c|}
\hline
Basis Function & Dimension & Polygon Type & Integration & Evaluation \\
\hline \hline
Wachspress	&2D/3D&	Convex&	Numerical	&Direct\\ \hline
PWL&	1D/2D/3D&	Convex/Concave&	Analytical	&Direct\\ \hline
Mean Value&	2D&	Convex/Concave&	Numerical	&Direct\\ \hline
Max Entropy&	1D/2D/3D	&Convex/Concave&	Numerical&	Iterative\\ \hline
\end{tabular}
\label{tab::2Dlin_summary}
\end{table}

%%%%%%%%%%%%%%%%%%%%%%%%%%%%%%%%%%%%%%%%%%%%%%%%%%%%%%%%%%%%%%%%%%%%
%%%%%%%%%%%%%%%%%%%%%%%%%%%%%%%%%%%%%%%%%%%%%%%%%%%%%%%%%%%%%%%%%%%%
\section{Quadratic Serendipity Polygonal Basis Functions} \label{sec::quadpoly}
%%%%%%%%%%%%%%%%%%%%%%%%%%%%%%%%%%%%%%%%%%%%%%%%%%%%%%%%%%%%%%%%%%%%
%%%%%%%%%%%%%%%%%%%%%%%%%%%%%%%%%%%%%%%%%%%%%%%%%%%%%%%%%%%%%%%%%%%%

\tcr
{
\begin{enumerate}
\item Image of conversion procedure
\item conversion methodology
\item example contour plots
\end{enumerate}
}

% serendipity overview figure
% ---------
\iffalse
\begin{figure}[hbt]
\centering
\includegraphics[width=\textwidth]{\BFfigpath{linear_to_quad_process.png}}
\caption{Overview of the process to construct the quadratic serendipity basis functions on polygons. The filled dots correspond to basis functions that maintain the Lagrange property while empty dots do not.}
\label{fig::BF_2D_quad_process}
\end{figure}
\fi
% ---------

%%%%%%%%%%%%%%%%%%%%%%%%%%%%%%%%%%%%%%%%%%%%%%%%%%%%%%%%%%%%%%%%%%%%
%%%%%%%%%%%%%%%%%%%%%%%%%%%%%%%%%%%%%%%%%%%%%%%%%%%%%%%%%%%%%%%%%%%%
\section{Adaptive Mesh Refinement Using Polygonal Basis Functions} \label{sec::amr}
%%%%%%%%%%%%%%%%%%%%%%%%%%%%%%%%%%%%%%%%%%%%%%%%%%%%%%%%%%%%%%%%%%%%
%%%%%%%%%%%%%%%%%%%%%%%%%%%%%%%%%%%%%%%%%%%%%%%%%%%%%%%%%%%%%%%%%%%%

\tcr{overview of AMR for transport calculations}

%%%%%%%%%%%%%%%%%%%%%%%%%%%%%%%%%%%%%%%%%%%%%%%%%%%%%%%%%%%%%%%%%%%%
%%%%%%%%%%%%%%%%%%%%%%%%%%%%%%%%%%%%%%%%%%%%%%%%%%%%%%%%%%%%%%%%%%%%
\section{Numerical Results} \label{sec::results}
%%%%%%%%%%%%%%%%%%%%%%%%%%%%%%%%%%%%%%%%%%%%%%%%%%%%%%%%%%%%%%%%%%%%
%%%%%%%%%%%%%%%%%%%%%%%%%%%%%%%%%%%%%%%%%%%%%%%%%%%%%%%%%%%%%%%%%%%%

%------------------------------------------------------------------------------------------------------------
\subsection{Exactly-Linear Transport Solutions}
%------------------------------------------------------------------------------------------------------------

%------------------------------------------------------------------------------------------------------------
\subsection{Exactly-Quadratic Transport Solutions}
%------------------------------------------------------------------------------------------------------------
\tcr{give table 1e-13 and the other one. mention why not plots}

%------------------------------------------------------------------------------------------------------------
\subsection{Convergence Rate Analysis by the Method of Manufactured Solutions}
%------------------------------------------------------------------------------------------------------------
\tcr{no mention of irr}

%------------------------------------------------------------------------------------------------------------
\subsection{Convergence Rate Analysis Bounded by the Solution Regularity}
%------------------------------------------------------------------------------------------------------------
\tcr{summarize results 
give poly only for cv rate=1/2. put the 3/2 in appendix.}

%------------------------------------------------------------------------------------------------------------
\subsection{Thick Diffusion Limit}
%------------------------------------------------------------------------------------------------------------
\tcr{\begin{itemize}
\item 1D unresolved boundary layer
\item 2D from dissertation
\end{itemize}}


%%%%%%%%%%%%%%%%%%%%%%%%%%%%%%%%%%%%%%%%%%%%%%%%%%%%%%%%%%%%%%%%%%%%
%%%%%%%%%%%%%%%%%%%%%%%%%%%%%%%%%%%%%%%%%%%%%%%%%%%%%%%%%%%%%%%%%%%%
\section{Conclusions} \label{sec::conclusions}
%%%%%%%%%%%%%%%%%%%%%%%%%%%%%%%%%%%%%%%%%%%%%%%%%%%%%%%%%%%%%%%%%%%%
%%%%%%%%%%%%%%%%%%%%%%%%%%%%%%%%%%%%%%%%%%%%%%%%%%%%%%%%%%%%%%%%%%%%


%%%%%%%%%%%%%%%%%%%%%%%%%%%%%%%%%%%%%%%%%%%%%%%%%%%%%%%%%%%%%%%%%%%%
%%%%%%%%%%%%%%%%%%%%%%%%%%%%%%%%%%%%%%%%%%%%%%%%%%%%%%%%%%%%%%%%%%%%
\section*{Acknowledgments} 
%%%%%%%%%%%%%%%%%%%%%%%%%%%%%%%%%%%%%%%%%%%%%%%%%%%%%%%%%%%%%%%%%%%%
%%%%%%%%%%%%%%%%%%%%%%%%%%%%%%%%%%%%%%%%%%%%%%%%%%%%%%%%%%%%%%%%%%%%
This research was performed under appointment to the Rickover Graduate Fellowship Program in Nuclear Engineering sponsored by the Naval Reactors Division of the United States Department of Energy.

%%%%%%%%%%%%%%%%%%%%%%%%%%%%%%%%%%%%%%%%%%%%%%%%%%%%%%%%%%%%%%%%%%%%
%%%%%%%%%%%%%%%%%%%%%%%%%%%%%%%%%%%%%%%%%%%%%%%%%%%%%%%%%%%%%%%%%%%%
%%%%%%%%%%%%%%%%%%%%%%%%%%%%%%%%%%%%%%%%%%%%%%%%%%%%%%%%%%%%%%%%%%%%
%%%%%%%%%%%%%%%%%%%%%%%%%%%%%%%%%%%%%%%%%%%%%%%%%%%%%%%%%%%%%%%%%%%%

\bibliographystyle{unsrt}
\bibliography{references}


%%%%%%%%%%%%%%%%%%%%%%%%%%%%%%%%%%%%%%%%%%%%%%%%%%%%%%%%%%%%%%%%%%%%
%%%%%%%%%%%%%%%%%%%%%%%%%%%%%%%%%%%%%%%%%%%%%%%%%%%%%%%%%%%%%%%%%%%%
\end{document}
%%%%%%%%%%%%%%%%%%%%%%%%%%%%%%%%%%%%%%%%%%%%%%%%%%%%%%%%%%%%%%%%%%%%
%%%%%%%%%%%%%%%%%%%%%%%%%%%%%%%%%%%%%%%%%%%%%%%%%%%%%%%%%%%%%%%%%%%%
