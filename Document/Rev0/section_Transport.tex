%%%%%%%%%%%%%%%%%%%%%%%%%%%%%%%%%%%%%%%%%%%%%%%%%%%
%
%  New template code for TAMU Theses and Dissertations starting Fall 2012.  
%  For more info about this template or the 
%  TAMU LaTeX User's Group, see http://www.howdy.me/.
%
%  Author: Wendy Lynn Turner 
%	 Version 1.0 
%  Last updated 8/5/2012
%
%%%%%%%%%%%%%%%%%%%%%%%%%%%%%%%%%%%%%%%%%%%%%%%%%%%
%%%                           SECTION II
%%%%%%%%%%%%%%%%%%%%%%%%%%%%%%%%%%%%%%%%%%%%%%%%%%%
\chapter{\uppercase {The Multigroup $S_N$ Equations}}



\begin{equation}
\label{eq::gen_boltzmann}
\frac{\partial u}{\partial t} = \left( \frac{\partial u}{\partial t}  \right)_{force} + \left( \frac{\partial u}{\partial t}  \right)_{advec} + \left( \frac{\partial u}{\partial t}  \right)_{coll}
\end{equation}

\noindent where $u(\vec{r},\vec{p},t)$ is the transport distribution function parameterized in terms of position, $\vec{r}=(x,y,z)$, momentum, $\vec{p}=(p_x,p_y,p_z)$, and time, $t$. The time rate of the change of the distribution function, $\frac{\partial u}{\partial t}$, is equal to the sum of the change rates due to external forces, $\left( \frac{\partial u}{\partial t}  \right)_{force} $, advection of the particles, $\left( \frac{\partial u}{\partial t}  \right)_{advec}$, and particle-to-particle and particle-to-matter collisions, $\left( \frac{\partial u}{\partial t}  \right)_{coll}$.

\begin{enumerate}
	\item Particles may be considered as points;
	\item Particles do not interact with other particles;
	\item Particles interact with material target atoms in a binary manner;
	\item Collisions between particles and material target atoms are instantaneous;
	\item Particles do not experience any external force fields ({\em e.g.} gravity).
\end{enumerate}

%%%%%%%%%%%%%%%%%%%%%%%%%%%%%%%%%%%%%%%%%%%%%%%%%%%
%%%   Section - Neutron Transport Equation
\section{The Neutron Transport Equation}
\label{sec::Sn_neut}

The neutron angular flux, $\Psi (\vec{r}, E, \vec{\Omega})$, at spatial position $\vec{r}$, with energy $E$ moving in direction $\vec{\Omega}$, is defined within an open, convex spatial domain $\mathcal{D}$, with boundary, $\partial \mathcal{D}$ by the general neutron transport equation:


\begin{equation}
\label{eq::transport_eq_full_source}
\begin{aligned}
	\vec{\Omega} \cdot \vec{\nabla} \Psi (\vec{r}, E, \vec{\Omega})+ \sigma_t (\vec{r}, E) \Psi (\vec{r}, E, \vec{\Omega}) = \frac{\chi (\vec{r}, E)}{4 \pi} \int dE' \nu \sigma_f (\vec{r}, E') \int d\Omega' \Psi (\vec{r}, E', \vec{\Omega}') \\ 
	+ \int dE' \int d\Omega' \sigma_s (E' \rightarrow E, \Omega' \rightarrow \Omega) \Psi (\vec{r}, E', \vec{\Omega}') + Q_{ext} (\vec{r}, E, \vec{\Omega})
\end{aligned}
\end{equation}

\noindent with the following, general boundary condition:

\begin{equation}
\label{eq::transport_bc_full}
\begin{aligned}
	\Psi (\vec{r}, E, \vec{\Omega}) = \Psi^{inc} (\vec{r}, E, \vec{\Omega}) + \int dE' \int d\Omega' \gamma (\vec{r}, E' \rightarrow E, \vec{\Omega}' \rightarrow \vec{\Omega}) \Psi (\vec{r}, E', \vec{\Omega}') \\
	\text{for } \vec{r} \in \partial \mathcal{D}^{-} \left\{   \partial \mathcal{D}, \vec{\Omega}' \cdot \vec{n} < 0  \right\}
\end{aligned} .
\end{equation}

\noindent In Eqs. (\ref{eq::transport_eq_full_source}) and (\ref{eq::transport_bc_full}), the physical properties of the system are defined as the following: $\sigma_t (\vec{r}, E)$ is the total neutron cross section, $\chi (\vec{r}, E)$ is the neutron fission spectrum, $\sigma_f (\vec{r}, E')$ is the fission cross section, $\nu (\vec{r}, E')$ is the average number of neutroncs emitted per fission, $\sigma_s (E' \rightarrow E, \Omega' \rightarrow \Omega)$ is the scattering cross section, and $Q_{ext} (\vec{r}, E, \vec{\Omega})$ is a distributed external source.


\begin{equation}
\label{eq::transport_eq_full_keff}
\begin{aligned}
	\vec{\Omega} \cdot \vec{\nabla} \Psi (\vec{r}, E, \vec{\Omega})+ \sigma_t (\vec{r}, E) \Psi (\vec{r}, E, \vec{\Omega}) = \frac{\chi (\vec{r}, E)}{4 \pi} \int dE' \nu \sigma_f (\vec{r}, E') \int d\Omega' \Psi (\vec{r}, E', \vec{\Omega}') \\ 
	+ \int dE' \int d\Omega' \sigma_s (E' \rightarrow E, \Omega' \rightarrow \Omega) \Psi (\vec{r}, E', \vec{\Omega}') 
\end{aligned}
\end{equation}

%%%%%%%%%%%%%%%%%%%%%%%%%%%%%%%%%%%%%%%%%%%%%%%%%%%
%%%   Section - Thermal Radiative Transfer Equations 
\section{The Thermal Radiative Transfer Equation}
\label{sec::Sn_TRT}

%%%%%%%%%%%%%%%%%%%%%%%%%%%%%%%%%%%%%%%%%%%%%%%%%%%
%%%   Section - Energy Discretization
\section{Energy Discretization}
\label{sec::Sn_MG}

%%%%%%%%%%%%%%%%%%%%%%%%%%%%%%%%%%%%%%%%%%%%%%%%%%%
%%%   Section - Angle Discretization
\section{Angular Discretization}
\label{sec::Sn_Angle}

%%%%%%%%%%%%%%%%%%%%%%%%%%%%%%%%%%%%%%%%%%%%%%%%%%%
%%%   Section - Spatial Discretization
\section{Spatial Discretization}
\label{sec::Sn_Spatial}


%%%%%%%%%%%%%%%%%%%%%%%%%%%%%%%%%%%%%%%%%%%%%%%%%%%
%%%   Section - Conclusions
\section{Conclusions}
\label{sec::Sn_Conclusions}