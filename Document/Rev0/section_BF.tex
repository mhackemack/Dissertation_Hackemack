%%%%%%%%%%%%%%%%%%%%%%%%%%%%%%%%%%%%%%%%%%%%%%%%%%%
%
%  New template code for TAMU Theses and Dissertations starting Fall 2012.  
%  For more info about this template or the 
%  TAMU LaTeX User's Group, see http://www.howdy.me/.
%
%  Author: Wendy Lynn Turner 
%	 Version 1.0 
%  Last updated 8/5/2012
%
%%%%%%%%%%%%%%%%%%%%%%%%%%%%%%%%%%%%%%%%%%%%%%%%%%%
%%%                           SECTION V
%%%%%%%%%%%%%%%%%%%%%%%%%%%%%%%%%%%%%%%%%%%%%%%%%%%
\chapter{\uppercase {FEM Basis Functions for Unstructured Polytopes}}
\label{sec::BF}

%%%%%%%%%%%%%%%%%%%%%%%%%%%%%%%%%%%%%%%%%%%%%%%%%%%
%%%   Section - 2D
\section{Two-Dimensional Basis Functions on Polygons}
\label{sec::BF_2D}


%%%%%%%%%%%%%%%%%%%%%%%%%%%%%%%%%%%%%%%%%%%%%%%%%%%
%%%   SubSection - Linear Basis Functions
\subsection{Linear Basis Functions}
\label{sec::BF_2D_Linear}

\begin{equation}
\begin{aligned}
	\sum_{i=1}^{N_K} b_i (\vec{x}) & =  1 \\
	\sum_{i=1}^{N_K} b_i(\vec{x}) \vec{x}_i & =  \vec{x}
\end{aligned}
\label{eq::linear_interp_requirements}
\end{equation}

%%%%%%%%%%%%%%%%%%%%%%%%%%%%%%%%%%%%%%%%%%%%%%%%%%%
%%%   SubSubSection - Linear
\subsubsection{Linear and BiLinear Basis Functions}
\label{sec::BF_2D_Linear_LDandBLD}

Before presenting basis function sets applicable to polytope finite elements, we first provide two basis functions that are exact 

\begin{equation}
\label{eq::2D_lin_basis_functions}
\begin{aligned}
	b_1(r,s) & = 1-r-s \\
	b_2(r,s) & = r \\
	b_3(r,s) & = s 
\end{aligned}
\end{equation}

and

\begin{equation}
\label{eq::3D_lin_basis_functions}
\begin{aligned}
	b_1(r,s) & = 1-r-s-t \\
	b_2(r,s) & = r \\
	b_3(r,s) & = s \\
	b_4(r,s) & = t
\end{aligned}
\end{equation}

%%%%%%%%%%%%%%%%%%%%%%%%%%%%%%%%%%%%%%%%%%%%%%%%%%%
%%%   SubSection - BiL/TriL
\subsection{BiLinear and TriLinear Basis Functions}
\label{sec::BF_Linear_BiLTriL}

Besides the linear basis functions of Section \ref{sec::BF_Linear_Linear}, we also present basis functions that are well defined on regular rectangular and parallelepiped spatial cells. These are the BiLinear (BiL) and TriLinear (TriL) basis functions for 2D and 3D, respectively. We demonstrate the BiL and TriL functional forms on reference elements consisting of the unit square and unit cube. These reference elements are presented in Figure \ref{fig::unit_square_cude_linear} and include the basis function numbers corresponding to their appropriate vertex location. For the unit square, we introduced the 2D reference coordinate system $(r,s)$, and for the unit cube, we introduced the 3D reference coordinate system $(r,s,t)$.

\begin{figure}
\centering
	\begin{subfigure}[b]{0.45\textwidth}
		\centering
		\label{subfig::unit_square}
		\includegraphics[width=\textwidth]{figures/sec_BF/unit_square_linear.png}
		\caption{}
	\end{subfigure}
	\hfill
	\begin{subfigure}[b]{0.45\textwidth}
		\centering
		\label{subfig::unit_cube}
		\includegraphics[width=\textwidth]{figures/sec_BF/unit_cube_linear.png}
		\caption{}
	\end{subfigure}
\caption{Vertex structure for the (a) unit square and (b) unit cube.}
\label{fig::unit_square_cude_linear}
\end{figure}

The 2D BiL basis functions on the reference square are:

\begin{equation}
\label{eq::BiL_basis_functions}
\begin{aligned}
	b_1(r,s) & = (1-r)(1-s) \\
	b_2(r,s) & = r(1-s) \\
	b_3(r,s) & = rs \\
	b_4(r,s) & = (1-r)s
\end{aligned}
\end{equation}

\noindent and

\begin{equation}
\label{eq::TriL_basis_functions}
\begin{aligned}
	b_1(r,s,t) & = (1-r)(1-s)(1-t) \\
	b_2(r,s,t) & = r(1-s)(1-t) \\
	b_3(r,s,t) & = rs(1-t) \\
	b_4(r,s,t) & = (1-r)s(1-t) \\
	b_5(r,s,t) & = (1-r)(1-s)t \\
	b_6(r,s,t) & = r(1-s)t \\
	b_7(r,s,t) & = rst \\
	b_8(r,s,t) & = (1-r)st \\
\end{aligned}
\end{equation}

%%%%%%%%%%%%%%%%%%%%%%%%%%%%%%%%%%%%%%%%%%%%%%%%%%%
%%%   SubSection - Wachspress
\subsection{Wachspress Rational Basis Functions}
\label{sec::BF_Linear_Wachspress}

%%%%%%%%%%%%%%%%%%%%%%%%%%%%%%%%%%%%%%%%%%%%%%%%%%%
%%%   SubSection - Mean Value
\subsection{Mean Value Basis Functions}
\label{sec::BF_Linear_MV}

%%%%%%%%%%%%%%%%%%%%%%%%%%%%%%%%%%%%%%%%%%%%%%%%%%%
%%%   SubSection - Metric
\subsection{Metric Basis Functions}
\label{sec::BF_Linear_Metric}

%%%%%%%%%%%%%%%%%%%%%%%%%%%%%%%%%%%%%%%%%%%%%%%%%%%
%%%   SubSection - Maximum Entropy
\subsection{Maximum Entropy Basis Functions}
\label{sec::BF_Linear_ME}

%%%%%%%%%%%%%%%%%%%%%%%%%%%%%%%%%%%%%%%%%%%%%%%%%%%
%%%   SubSection - PWL
\subsection{Piecewise Linear (PWL) Basis Functions}
\label{sec::BF_Linear_PWL}

\begin{equation}
\label{eq::PWL_2D}
	b_j (x,y) = t_j (x,y) + \alpha_j^K t_c (x,y)
\end{equation}

\noindent $t_j$ is the standard 2D linear function with unity at vertex $j$ that linearly decreases to zero to the cell center and each adjoining vertex. $t_c$ is the 2D cell ``tent'' function which is unity at the cell center and linearly decreases to zero to each cell vertex. $\alpha_{K,j}$ is the weight parameter for vertex $j$ in cell $K$.

The 3D PWL basis functions share a similar form to the 2D PWL basis functions.

\begin{equation}
\label{eq::PWL_3D}
	b_j (x,y,z)  = t_j  (x,y,z) + \sum_{f=1}^{F_j} \beta_j^f  t_f (x,y,z) + \alpha_j^K t_c  (x,y,z)
\end{equation}

\noindent $t_j$ is the standard 3D linear function with unity at vertex $j$ that linearly decreases to zero to the cell center, the face center for each face that includes vertex $j$, and each vertex that shares an edge with vertex $j$. $t_c$ is the 3D cell ``tent" function which is unity at the cell center and linearly decreases to zero to each cell vertex and face center. $t_f$ is the face "tent" function which is unity at the face center and linearly decreases to zero at each vertex on that face and the cell center. $\beta_{f,j}$ is the weight parameter for face $f$ touching cell vertex $j$, and $F_j$ is the number of faces touching vertex $j$. Like the previous work defining the PWLD method \cite{bailey2008phd}, we also choose to assume the cell and face weighting parameters are

\begin{equation}
\alpha_{K,j} = \frac{1}{N_K} \qquad \text{and} \qquad \beta_{f,j} = \frac{1}{N_f},
\label{eq::PWL_weight_vals}
\end{equation}

\noindent respectively, where $N_K$ is the number of vertices in cell $K$ and $N_f$ is the number of vertices on face $f$, which leads to constant values of $\alpha$ and $\beta$ for each cell and face, respectively. This assumption of the cell weight function holds for both 2D and 3D.

%%%%%%%%%%%%%%%
% Begin::2D PWL basis function plots
\pagebreak
\begin{figure}
\label{fig::2D_PWL_unit_square_basis_functions}
\centering
	\begin{subfigure}[b]{0.48\textwidth}
		\centering
		\includegraphics[width=\textwidth]{figures/sec_BF/PWL_square_contour_1.png}
		\caption{}
	\end{subfigure}
	\hfill
	\begin{subfigure}[b]{0.48\textwidth}
		\centering
		\includegraphics[width=\textwidth]{figures/sec_BF/PWL_square_contour_2.png}
		\caption{}
	\end{subfigure}
	\vfill
	\begin{subfigure}[b]{0.48\textwidth}
		\centering
		\includegraphics[width=\textwidth]{figures/sec_BF/PWL_square_contour_3.png}
		\caption{}
	\end{subfigure}
	\hfill
	\begin{subfigure}[b]{0.48\textwidth}
		\centering
		\includegraphics[width=\textwidth]{figures/sec_BF/PWL_square_contour_4.png}
		\caption{}
	\end{subfigure}
\caption{Contour plots of the PWL basis functions on the unit square for the vertices located at: (a) (0,0), (b) (1,0), (c) (1,1), and (d) (0,1).}
\end{figure}

\begin{figure}
\label{fig::2D_pentagon_vertices}
\centering
	\begin{subfigure}[b]{0.40\textwidth}
		\centering
		\includegraphics[width=\textwidth]{figures/sec_BF/reg_pent_verts.png}
		\caption{}
	\end{subfigure}
	\hfill
	\begin{subfigure}[b]{0.40\textwidth}
		\centering
		\includegraphics[width=\textwidth]{figures/sec_BF/deg_pent_verts.png}
		\caption{}
	\end{subfigure}
\caption{Vertex structure for a (a) regular pentagonal cell and a (b) degenerate pentagonal cell.}
\end{figure}

\begin{figure}
\label{fig::2D_PWL_pentagon_basis_functions_contour}
\centering
	\begin{subfigure}[b]{0.48\textwidth}
		\centering
		\includegraphics[width=\textwidth]{figures/sec_BF/PWL_rpent_contour_A.png}
		\caption{}
	\end{subfigure}
	\hfill
	\begin{subfigure}[b]{0.48\textwidth}
		\centering
		\includegraphics[width=\textwidth]{figures/sec_BF/PWL_dpent_contour_A.png}
		\caption{}
	\end{subfigure}
	\vfill
	\begin{subfigure}[b]{0.48\textwidth}
		\centering
		\includegraphics[width=\textwidth]{figures/sec_BF/PWL_rpent_contour_E.png}
		\caption{}
	\end{subfigure}
	\hfill
	\begin{subfigure}[b]{0.48\textwidth}
		\centering
		\includegraphics[width=\textwidth]{figures/sec_BF/PWL_dpent_contour_E.png}
		\caption{}
	\end{subfigure}
\caption{Contour plots of the PWL basis functions for a regular pentagon: (a) and (c) as well as a degenerate pentagon: (b) and (d).}
\end{figure}

\begin{figure}
\label{fig::2D_PWL_pentagon_basis_functions_plot}
\centering
	\begin{subfigure}[b]{0.48\textwidth}
		\centering
		\includegraphics[width=\textwidth]{figures/sec_BF/PWL_rpent_plot_A.png}
		\caption{}
	\end{subfigure}
	\hfill
	\begin{subfigure}[b]{0.48\textwidth}
		\centering
		\includegraphics[width=\textwidth]{figures/sec_BF/PWL_dpent_plot_A.png}
		\caption{}
	\end{subfigure}
	\vfill
	\begin{subfigure}[b]{0.48\textwidth}
		\centering
		\includegraphics[width=\textwidth]{figures/sec_BF/PWL_rpent_plot_E.png}
		\caption{}
	\end{subfigure}
	\hfill
	\begin{subfigure}[b]{0.48\textwidth}
		\centering
		\includegraphics[width=\textwidth]{figures/sec_BF/PWL_dpent_plot_E.png}
		\caption{}
	\end{subfigure}
\caption{Plots of the PWL basis functions for a regular pentagon: (a) and (c) as well as a degenerate pentagon: (b) and (d).}
\end{figure}


% End::2D PWL basis function plots
%%%%%%%%%%%%%%%

%%%%%%%%%%%%%%%%%%%%%%%%%%%%%%%%%%%%%%%%%%%%%%%%%%%
%%%   Section - Quadratic Basis Functions
\section{Quadratic Basis Functions}
\label{sec::BF_Quadratic}

\begin{gather}
	 \sum_{i=1}^{N_K} b_i (\vec{x})  =  1 \\
	\sum_{i=1}^{N_K} b_i(\vec{x}) \vec{x}_i  =  \vec{x} \\
	\sum_{i=1}^{N_K} \sum_{j=1}^{N_K} \mu _{i,j} b_i(\vec{x}) \left(   \frac{\vec{x}_i \otimes \vec{x}_j +\vec{x}_j \otimes \vec{x}_i }{2}  \right)  = \vec{x} \otimes \vec{x}
\label{eq::quadratic_interp_requirements}
\end{gather}

\noindent where $\mu_{i,j}$ is a weight function corresponding to particular basis function pairings.

%%%%%%%%%%%%%%%%%%%%%%%%%%%%%%%%%%%%%%%%%%%%%%%%%%%
%%%   SubSection - BiL/TriL
\subsection{Serendipity Bilinear and Trilinear Basis Functions}
\label{sec::BF_Quadratic_BiLTriL}

%%%%%%%%%%%%%%%%%%%%%%%%%%%%%%%%%%%%%%%%%%%%%%%%%%%
%%%   Section - 3D
\section{Three-Dimensional Basis Functions on Polyhedra}
\label{sec::BF_3D}

%%%%%%%%%%%%%%%%%%%%%%%%%%%%%%%%%%%%%%%%%%%%%%%%%%%
%%%   Section - Conclusions
\section{Conclusions}
\label{sec::BF_Conclusions}









