%%%%%%%%%%%%%%%%%%%%%%%%%%%%%%%%%%%%%%%%%%%%%%%%%%%
%
%  New template code for TAMU Theses and Dissertations starting Fall 2012.  
%  For more info about this template or the 
%  TAMU LaTeX User's Group, see http://www.howdy.me/.
%
%  Author: Wendy Lynn Turner 
%	 Version 1.0 
%  Last updated 8/5/2012
%
%%%%%%%%%%%%%%%%%%%%%%%%%%%%%%%%%%%%%%%%%%%%%%%%%%%
%%%                           SECTION III
%%%%%%%%%%%%%%%%%%%%%%%%%%%%%%%%%%%%%%%%%%%%%%%%%%%
\chapter{\uppercase{Massively Parallel Transport Sweeping on Unstructured Grids}}


%%%%%%%%%%%%%%%%%%%%%%%%%%%%%%%%%%%%%%%%%%%%%%%%%%%
%%%   Section - Polygonal/Polyhedral Grids
\section{Use of Polygonal and Polyhedral Grids}
\label{sec::Sweeps_Poly_Grids}

\begin{enumerate}
	\item Other physics communities are now employing polyhedral grids, which can increase the complexity of mesh mapping between different physics packages;
	\item Polygonal and Polyhedral grids can provide a better domain partition, yielding reduced cell and face counts (should reduce cpu run times);
	\item ({\em e.g.} boundary layers);
	\item Independently-generated simplicial grids ({\em i.e.} created in parallel) can be stitched together with polygons and polyhedra without communicating the whole mesh across processors;
	\item Hanging nodes from non-conforming meshes are not necessary (arise naturally in adaptive mesh refinement).
\end{enumerate}



%%%%%%%%%%%%%%%%%%%%%%%%%%%%%%%%%%%%%%%%%%%%%%%%%%%
%%%   Section - Conclusions
\section{Conclusions}
\label{sec::Sweeps_Conclusions}

We have presented
